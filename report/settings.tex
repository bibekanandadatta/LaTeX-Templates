%%%%%%%%%%% list of variables for different formatting settings %%%%%%%%%%%%%%%%

% make all of your changes here in the variables 
% before tweaking original settings down below

\def\NoSectionLevel{3}              % no of levels for sections ... to subsubsection
\def\NoTocLevel{3}                  % no of levels showed in the table of contents
\def\TocIndent{0}                   % indentation in the list of figs and tables


\def\ChapterTopSpace{-48}           % white space on top of the chapter heading
\def\ChapterToTitle{-12}            % space between chapter to title
\def\TitleToText{18}                % space between the chapter title to the following text

% font format for chapter heading and title
\def\ChapterFont{\singlespacing \Large \bfseries}
\def\SectionFont{\large\bfseries}           % section heading font format
\def\SubsectionFont{\normalsize\bfseries}   % subsection heading font format
\def\SubsubsectionFont{\normalsize\itshape} % subsubsection heading font format
\def\CaptionFontSize{small}         %caption font size
\def\CaptionFontType{bf}            % boldface font for captions
\def\CaptionSeparator{colon}        % separates caption heading from text. can use 'period' as well
\def\CodeFont{\footnotesize\ttfamily} % font for including codes

\def\QuoteWidth{0.65\textwidth}     % width of quote in epigraph
\def\GlobalTableSpacing{1.5}        % global spacing parameter for table (double spaced)
\def\ParagraphSpacing{\baselineskip}% spacing between paragraph
\def\ParagraphIndent{0}             % indentation at the beginning of the paragraph
\def\FullCiteSpacing{1.25}          % spacing in a fullcite item
\def\BibItemSpacing{0.5\baselineskip}  % spacing between bibliographic items in reference
\def\FootnoteSpacing{0.75\baselineskip} % spacing between footnotes
\def\CaptionSpacing{0}              % spacing between the figure and the caption (unit: pt)


\def\FontPackage{lmodern}           % latin modern font (you can also change it to times)
\def\BibFileName{template.bib}        % name of BibLaTeX file with all the bibliography
\def\FigurePath{figures}            % subdirectory for the figure files

%%%%%%%%%%% list of variables for different formatting settings %%%%%%%%%%%%%%%


%% if possible, make all of your changes here in the variables. you should be able
%% to make most common changes here before tweaking original settings down below.

%%%%%%%%%%%%%%%%%%%%%%%%%%%%%%%% LaTeX Packages %%%%%%%%%%%%%%%%%%%%%%%%%%%%%%%

%% add packages as you need but remember sometimes the order of the packages matter

\usepackage[utf8]{inputenc}	    %input
\DeclareUnicodeCharacter{2212}{-}


%% math packages
\usepackage{amsfonts,amssymb,amsmath,amsthm,dsfont,mathtools,mathbbol,siunitx,upgreek,mathrsfs,cancel}
\usepackage{autobreak}
\usepackage[ruled]{algorithm2e} %algorithm package
\usepackage[titletoc]{appendix}
\usepackage[american]{babel}


% bibliographic package (make sure your bib file is in BibLaTeX format)
% use Zotero or some other reference manager to generate the BibLaTeX file
% change the style or other options as needed
% \usepackage[backend=biber, style=apa, isbn=false, url=false, doi=true]{biblatex}
\usepackage[backend=biber, style=nature, maxnames=9, date=year, isbn=false, url=false, doi=true]{biblatex}

\usepackage{blindtext}

\usepackage{calc}
\usepackage{caption}
\usepackage{color,xcolor}
\usepackage{epigraph,varwidth}
\usepackage{enumitem}			%list environment
\usepackage{float}
\usepackage[T1]{fontenc}
\usepackage{graphicx}


\usepackage{geometry}
\usepackage{fancyhdr}
\usepackage[pdfa]{hyperref}
\usepackage[all]{hypcap}                    % for captions on the side of figures
\usepackage{ifthen}
\usepackage{lscape}
\usepackage[pagewise,mathlines]{lineno}     % linenumbers
\usepackage{csquotes}                       % for quote environment


\usepackage{listings}                       % to include codes
% table related packages
\usepackage{booktabs,longtable,makecell,multicol,multirow,tabularx,xltabular}
%\usepackage[protrusion]{microtype}
\usepackage{textcomp}
% \usepackage{sectsty}


\usepackage{setspace}
\usepackage{seqsplit}
\usepackage[rightcaption]{sidecap}
\usepackage{soul}
\usepackage[titles]{tocloft}
\usepackage{parskip}
\usepackage{titlesec}
%\usepackage{tocbibind}
\usepackage{tikz}
\usetikzlibrary{positioning}
\usetikzlibrary{shapes,arrows}
\usepackage{subcaption}    % Individual panel and caption


%% add or change more packages and options as you need

%%%%%%%%%%%%%%%%%%%%%%%%%%%%%%%% LaTeX Packages %%%%%%%%%%%%%%%%%%%%%%%%%%%%%%%




%%%%%%%%%%%%%%%%%%%%%%%%%%%%%%%%%%%%%%%%%%%%%%%%%%%%%%%%%%%%%%%%%%%%%%%%%%%%%%%
%%%%%%%%%%%%%%%%%%%%%%%%%%%%%%%%%%%%%%%%%%%%%%%%%%%%%%%%%%%%%%%%%%%%%%%%%%%%%%%

%% if you are unhappy about some specific formatting and can not change them by
%% changing defined variables in the beginning, only then proceed to the next 
%% sections which includes loading proper package options, redefining different
%% segments, document formatting, settings for special packages (epigraph, 
%% algorithm, listings, etc). please be cautious before making any changes.

%%%%%%%%%%%%%%%%%%%%%%%%%%%%%% PACKAGE OPTIONS %%%%%%%%%%%%%%%%%%%%%%%%%%%%%%%%

%% add all the images to that folder for cleaner file management
%% you can add images in chapter-wise PDF format (my preference).
\graphicspath{{\FigurePath/}}


%% this file has to be in BibLaTeX format. Use Zotero or some other citation manager to generate the .bib file in BibLaTeX format.
\addbibresource{\BibFileName}

%% margin settings required by JH library (geometry package)
%% if you have a long chapter title you may need to customize the header settings here
\geometry{letterpaper, left=1.0in, right=1.0in, top=1.0in, bottom=1.0in, includehead, headheight=15pt, headsep=10pt, includefoot, heightrounded}



%% settings for the hyperref package
\hypersetup{linktocpage, unicode, colorlinks=true, citecolor=blue, filecolor=blue, linkcolor=blue, urlcolor=blue}
% add 'linktoc=all' option to the above list for making the items in the 
% table of contents as clickable links


%% settings for caption package (customize or add more as needed)
\captionsetup{belowskip=\CaptionSpacing pt, font=\CaptionFontSize, labelfont=\CaptionFontType, labelsep=\CaptionSeparator, hypcap=true} 


%% settings for listing package to include code
\lstset{basicstyle=\CodeFont,columns=flexible,breaklines=true}

%%%%%%%%%%%%%%%%%%%%%%%%%%%%%% PACKAGE OPTIONS %%%%%%%%%%%%%%%%%%%%%%%%%%%%%%%%%


%%%%%%%%%%%%%%%%%%%%%%%%%%%%%%% REDEFINITION %%%%%%%%%%%%%%%%%%%%%%%%%%%%%%%%%%%

%%%% UNNUMBERED CHAPTERS, SECTION, and SUBSECTION COMMAND for ADDING to TOC
% removes the 'Chapter #' title while keeping it listed in the TOC
\newcommand\chap[1]{%
  \chapter*{#1}%
  \markboth{#1}{}
  \addcontentsline{toc}{chapter}{#1}}
  
% removes the 'Section #' title while keeping it listed in the TOC
\newcommand\sect[1]{%
  \section*{#1}%
  \addcontentsline{toc}{section}{#1}}
  
% Removes the 'Subsection #' title while keeping it listed in the TOC
\newcommand\subsect[1]{%
  \subsection*{#1}%
  \addcontentsline{toc}{subsection}{#1}}

% Removes the 'Subsubsection #' title while keeping it listed in the TOC
\newcommand\subsubsect[1]{%
  \subsubsection*{#1}%
  \addcontentsline{toc}{subsubsection}{#1}}

%%%%%%%%%%%%%%%%%%%%%%%%%%%%%%% REDEFINITION %%%%%%%%%%%%%%%%%%%%%%%%%%%%%%%%%%%



%%%%%%%%%%%%%%%%%%%%%%%%%%%%%%% SETTINGS FOR TOC %%%%%%%%%%%%%%%%%%%%%%%%%%%%%%%

%%%% TOC shows (chapter to subsection) in the list
\setcounter{tocdepth}{\NoTocLevel}
\setcounter{secnumdepth}{\NoSectionLevel}       % section to ... subsubsection

\setlength{\cftfigindent}{\TocIndent pt}        % indentation from figures in lof
\setlength{\cfttabindent}{\TocIndent pt}        % indentation from tables in lot


% dots for chapters too
\renewcommand{\cftchapleader}{\cftdotfill{\cftdotsep}}


% tweak to TOC to add 'chapter' to the chapter name instead of a number only
% set the width of the box based on the longest label name
\renewcommand{\cftchappresnum}{\chaptername\space}
\setlength{\cftchapnumwidth}{\widthof{\textbf{Appendix~999~}}}


% tweak to TOC to add 'Figure' to the figure caption listing
% to change the distance to the start of the figure title
\renewcommand{\cftfigpresnum}{\bfseries Figure }
\setlength{\cftfignumwidth}{\widthof{\textbf{Figure~99.999~}}}


% tweak to TOC to add 'Table' to the Table caption listing
% to change the distance to the start of the figure title
\renewcommand{\cfttabpresnum}{\bfseries Table }
\setlength{\cfttabnumwidth}{\widthof{\textbf{Table~99.100~}}}

%%%%%%%%%%%%%%%%%%%%%%%%%%%%%%% SETTINGS FOR TOC %%%%%%%%%%%%%%%%%%%%%%%%%%%%%%%



%%%%%%%%%%%%%%%%%%%%%%%%%%%%% DOCUMENT FORMATTING %%%%%%%%%%%%%%%%%%%%%%%%%%%%%%


%% choice a font form (or add something else) for your thesis (uncomment one option)
\usepackage{\FontPackage}       
% if you want to use Palatino font, use the following and comment above line
%\usepackage[sc]{mathpazo}      % palatino font family


%%%% chapter # and title settings (with gaps)
%% if you use an epigraph after the chapter title then perhaps reduce the first \vspace* from 0 pt to -(some_value)
\makeatletter
\def\@makechapterhead#1{%
  \vspace*{\ChapterTopSpace \p@}   % white space before the chapter #
  {\parindent \z@ \raggedright \normalfont
    \ifnum \c@secnumdepth >\m@ne
        \ChapterFont \@chapapp\enskip \thechapter 
        \\ \vspace{\ChapterToTitle \p@}   % space between chapter # and title
    \fi
        \interlinepenalty\@M
        \ChapterFont #1\par\nobreak
    \vskip \TitleToText \p@         % space between the chapter title and the following text
  }}
\makeatother


%%%% settings for unnumbered chapters
% if you use an epigraph after the chapter title then perhaps reduce the first \vspace* from 0 pt to - (some_value)
\makeatletter
\def\@makeschapterhead#1{%
  \vspace*{\ChapterTopSpace \p@} % white space before to the chapter title
  {\parindent \z@ \raggedright
    \normalfont
    \interlinepenalty\@M
    \ChapterFont  #1\par\nobreak
    \vskip \TitleToText \p@     % space between the chapter title and the following text
  }}
\makeatother



%%%% using titlesec package for sections, subsection, ... heading format
\titleformat*{\section}{\SectionFont}
\titleformat*{\subsection}{\SubsectionFont}
\titleformat*{\subsubsection}{\SubsubsectionFont}


%%%% settings for paragraph (and not title) spacing, roughly speaking
\renewcommand{\arraystretch}{\GlobalTableSpacing}   % spacing inside table
\setlength{\parskip}{\ParagraphSpacing}             % paragraph skip
\setlength{\parindent}{\ParagraphIndent pt}         % paragraph indentation
\setlength{\bibitemsep}{\BibItemSpacing}            % bib item separation 
\setlength{\footnotesep}{\FootnoteSpacing}          % separation between footnote


%%%% bibliography package settings
\DeclareFieldFormat{titlecase}{\MakeSentenceCase*{#1}}
\AtBeginBibliography{\urlstyle{rm}}
\DeclareBibliographyCategory{fullcited}
\newcommand{\mybibexclude}[1]{\addtocategory{fullcited}{#1}}


%%%%%%%%%%%%%%%%%%%%%%%%%%% END DOCUMENT FORMATTING %%%%%%%%%%%%%%%%%%%%%%%%%%%


%%%%%%%%%%%%%%%%%%%%%%%%%%%%% EPIGRAPH SETTINGS %%%%%%%%%%%%%%%%%%%%%%%%%%%%%%%

%%% Following settings allow the epigraph and the underline 
%%% settings for arbitrary epigraph length
\renewcommand{\epigraphflush}{flushright}
\renewcommand{\epigraphsize}{\small}
\setlength{\epigraphwidth}{\QuoteWidth}
\renewcommand{\textflush}{flushright}
\renewcommand{\sourceflush}{flushright}
% A useful addition
\newcommand{\epitextfont}{\itshape}
\newcommand{\episourcefont}{\scshape}

\makeatletter
\newsavebox{\epi@textbox}
\newsavebox{\epi@sourcebox}
\newlength\epi@finalwidth
\renewcommand{\epigraph}[2]{%
  \vspace{\beforeepigraphskip}
  {\epigraphsize\begin{\epigraphflush}
   \epi@finalwidth=\z@
   \sbox\epi@textbox{%
     \varwidth{\epigraphwidth}
     \begin{\textflush}\epitextfont#1\end{\textflush}
     \endvarwidth
   }%
   \epi@finalwidth=\wd\epi@textbox
   \sbox\epi@sourcebox{%
     \varwidth{\epigraphwidth}
     \begin{\sourceflush}\episourcefont#2\end{\sourceflush}%
     \endvarwidth
   }%
   \ifdim\wd\epi@sourcebox>\epi@finalwidth 
     \epi@finalwidth=\wd\epi@sourcebox
   \fi
   \leavevmode\vbox{
     \hb@xt@\epi@finalwidth{\hfil\box\epi@textbox}
     \vskip 1ex         % gap between quote and rule
     \hrule height \epigraphrule
     \vskip 1ex         % gap between rule and author
     \hb@xt@\epi@finalwidth{\hfil\box\epi@sourcebox}
   }%
   \end{\epigraphflush}
   \vspace{\afterepigraphskip}}}
\makeatother

%%%%%%%%%%%%%%%%%%%%%%%%%%%%% EPIGRAPH SETTINGS %%%%%%%%%%%%%%%%%%%%%%%%%%%%%%%



%%%%%%%%%%%%%%%%%%%%%% ALGORITHM AND LISTING SETTINGS %%%%%%%%%%%%%%%%%%%%%%%%%

%% settings for algorithm2e package
\renewcommand{\algorithmcfname}{Procedure}
\SetKwFor{While}{while}{}{end while}%
\SetArgSty{textnormal}
\newcommand\mycommfont[1]{\footnotesize\ttfamily\textcolor{blue}{#1}}
\SetCommentSty{mycommfont}



%% listing package definition (to add code in the document)
\usepackage{listings}
\lstdefinestyle{terminal}{columns=fullflexible,
keepspaces=true,
breaklines=true,
basicstyle={\footnotesize\fontfamily{fvm}\fontseries{m}\selectfont},
keywordstyle={\footnotesize\fontfamily{fvm}\fontseries{b}\selectfont},
commentstyle={\color{comments}\small\fontfamily{fvm}\itshape\selectfont},
frame=single,
xleftmargin=0in,
backgroundcolor=\color{lightgray!50},
belowcaptionskip=10pt,
aboveskip=0.5cm}
\lstset{style=terminal,float=h,language=bash}

%%%%%%%%%%%%%%%%%%%% END ALGORITHM AND LISTING SETTINGS %%%%%%%%%%%%%%%%%%%%%%%



%%%%%%%%%%%%%%%%%%%%%%%%%%%%%%%%%%%%%%%%%%%%%%%%%%%%%%%%%%%%%%%%%%%%%%%%%%%%%%%
%%%%%%%%%%%%%%%%%%%%%%%%%%%%%%%%%%%%%%%%%%%%%%%%%%%%%%%%%%%%%%%%%%%%%%%%%%%%%%%




%%%%%%%%%%%%%%%%%%%%%%%%%%%%%%%%%%%%%%%%%%%%%%%%%%%%%%%%%%%%%%%%%%%%%%%%%%%%%%%
%% add all your custom math settings and macros in the following section

%%%%%%%%%%%%%%%%%%%%%%% MATH SETTINGS AND MACROS %%%%%%%%%%%%%%%%%%%%%%%%%%%%%%

\numberwithin{equation}{chapter}        % eqn no with chapter prefix

\allowdisplaybreaks[1]
\setcounter{MaxMatrixCols}{20}

% define math symbols
\numberwithin{equation}{section}
\newcommand{\dC}{$^{\circ}$C}
\DeclareMathOperator{\R}{\mathrm{R}}
\DeclareMathOperator{\T}{{\top}}
\newcommand{\vect}[1]{\mathbf{#1}}

\newcommand{\mat}[1]{\mathbf{#1}}
\DeclareMathOperator{\tr}{tr}
\DeclareMathOperator{\sym}{sym}
\DeclareMathOperator{\skw}{skw}

\DeclareMathOperator{\divg}{div}
\DeclareMathOperator{\grad}{grad}
\DeclareMathOperator{\curl}{curl}
\DeclareMathOperator{\Grad}{Grad}
\DeclareMathOperator{\Divg}{Div}
\DeclareMathOperator{\Curl}{Curl}

\DeclareMathOperator{\sgn}{sign}


\DeclareMathOperator*{\Aop}{A}
\makeatletter
\DeclareRobustCommand\bigop[2][1]{%
  \mathop{\vphantom{\sum}\mathpalette\bigop@{{#1}{#2}}}\slimits@
}
\newcommand{\bigop@}[2]{\bigop@@#1#2}
\newcommand{\bigop@@}[3]{%
  \vcenter{%
    \sbox\z@{$#1\sum$}%
    \hbox{\resizebox{\ifx#1\displaystyle#2\fi\dimexpr\ht\z@+\dp\z@}{!}{$\m@th#3$}}%
  }%
}
\makeatother
\newcommand{\bigA}{\DOTSB\bigop[1]{\mathsf{A}}}
\theoremstyle{definition}
\newtheorem{remark}{Remark}

%%%%%%%%%%%%%%%%%%%%%%% MATH SETTINGS AND MACROS %%%%%%%%%%%%%%%%%%%%%%%%%%%%%%



%%%%%%%%%%%%%%%%%%%%%%%%%%%%%% OTHER MACROS %%%%%%%%%%%%%%%%%%%%%%%%%%%%%%%%%%%

\newcommand{\COMMENT}{\textcolor{red}}
\newcommand{\ADDCITATION}{\COMMENT{(ADD CITATION)}}
\newcommand{\ADDFIGURE}{\COMMENT{(ADD FIGURE)}}
\newcommand{\ADDTABLE}{\COMMENT{(ADD TABLE)}}

%% you can also add more simple comments here as you need
%% you can use some other packages for more complicated review and comment section


%%%%%%%%%%%%%%%%%%%%%%%%%%%%%% OTHER MACROS %%%%%%%%%%%%%%%%%%%%%%%%%%%%%%%%%%%