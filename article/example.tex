%%%%%%%%%%%%%%%%%%%%%%%%%%%%%%%%%%%%%%%%%%%%%%%%%%%%%%%%%%%%%%%%%%%%%%%%%
%%%%%%%%%%%%%%%%%%%%     SOME EXAMPLE ENVIRONMENTS    %%%%%%%%%%%%%%%%%%%
%%%%%%%%%%%%%%%%%%%%%%%%%%%%%%%%%%%%%%%%%%%%%%%%%%%%%%%%%%%%%%%%%%%%%%%%%

\begin{comment}

%% Check out this tutorial for list: https://www.overleaf.com/learn/latex/Lists

%% Produces an un-numbered list of items with 0.75 cm distance from the left margin with line spacing between items and top with diamond label
\begin{itemize}[leftmargin=0.75cm, itemsep=-0.3cm, topsep=-0.3cm,label={$\bullet$}]
    \item 
    \item
    \item
\end{itemize}

%% Produces numbered item lists
\begin{enumerate}[leftmargin=0.75cm, itemsep=-0.3cm, topsep=-0.3cm]
  \item 
  \item 
  \item 
\end{enumerate}

%% Produces a nested itemized list
\begin{enumerate}
    \item 
    \item 
        \begin{description}[topsep=5pt, style=multiline, leftmargin= 2.5cm, itemsep = 5pt]
            \item [Research] 
                \begin{itemize}
                    \item 
                    \item
                \end{itemize}
            \item [Academic]
                \begin{enumerate}
                    \item 
                    \item
                \end{enumerate}
        \end{description}
    \item 
\end{enumerate}

%% HIGHLY RECOMMENDED: Check out these tutorials for math: https://www.overleaf.com/learn/latex/Mathematics

%% Produces numbered equations
\begin{equation} \label{eq:some-name}
    % some equation
\end{equation}

%% Produces a centered set of equations
\begin{equation} \label{eq:some-name}
    \begin{gathered}
        % some equation \\
        % next equation \\
        % another equation
    \end{gathered}
\end{equation}

%% Produces aligned multi-line equations (aligned along the & symbols)
\begin{equation} \label{eq:some-name}
    \begin{aligned}
    % some & equation \\
    % next & equation \\
    % another & equation
    \end{aligned}
\end{equation}

%% Adds figure of specific width (height is scaled but you can define height and angle as well)
\begin{figure}[ht]
\begin{center}
    \includegraphics[width= a cm] {filename.extension}
    \caption{some-caption}
    \label{fig:some-name}
\end{center}
\end{figure}

%% Adds figure from a specific page of a pdf. It specifies the width (a) of the picture in the document and trims from the left bottom right top for b, c, d, e cm.
\begin{figure}[ht]
\begin{center}
    \includegraphics[width= a cm, trim={b cm c cm d cm e cm}, clip, page= f] {some-pdf.pdf}
    \caption{some-caption}
    \label{fig:some-name}
\end{center}
\end{figure}

%% For the table, see the tutorial here: https://www.overleaf.com/learn/latex/Tables
%% Produces a simple two-column table w/ multiple rows
\begingroup
\begin{table} [ht]
\begin{center}
\renewcommand{\arraystretch}{1.25}
\begin{tabular}{c c} 
\hline \hline
\textbf{Header 1} & \textbf{Header 2} \\
\hline \hline
 & \\
 & \\
 & \\
\hline
\end{tabular}
\caption{some-caption}
\end{center}
\end{table}
\endgroup

%% Double minipage half the size of the page
\begin{minipage}[t]{0.5\textwidth}
	\begin{flushleft}
        \\
        \\
	\end{flushleft}
\end{minipage}
~
\begin{minipage}[t]{0.5\textwidth}
	\begin{flushleft}
        \\
        \\
	\end{flushleft}
\end{minipage}
\end{comment}

%%%%%%%%%%%%%%%%%%%%%%%%%%%%%%%%%%%%%%%%%%%%%%%%%%%%%%%%%%%%%%%%%%%%%%%%%%%%%